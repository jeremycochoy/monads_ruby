\documentclass{beamer}

\usepackage[frenchb]{babel}
\usepackage[T1]{fontenc}
\usepackage[utf8]{inputenc}

\usetheme{Warsaw}
\useoutertheme{infolines}

\usepackage{amsmath}
\usepackage{amssymb}
\usepackage{amsthm}
\usepackage{stmaryrd}

\usepackage[all]{xy}

\setbeamercovered{transparent}

\begin{document}

\title{Monades, Comonades et Automates cellulaires}
\author{Jérémy S. Cochoy}
\institute{INRIA Paris-Saclay}
\date{Octobre 2015}


\begin{frame}
\titlepage
\end{frame}

\begin{frame}
\tableofcontents
\end{frame}

\begin{frame}

\begin{center}
\includegraphics[scale=0.3]{screen1.png}
\end{center}

\end{frame}

\section{Monades}

\subsection{Types}
\begin{frame}
\frametitle{Les types}
\begin{block}{Qu'est-ce qu'un type?}
C'est un \emph{ensemble} de valeurs.
\end{block}
\pause
\begin{exampleblock}{Examples :}
\begin{itemize}
\item $Int = \{-2 147 483 648, \dots, 2 147 483 647\}$
\item $Bool = \{True, False\}$
\item $Char = \{'a', 'b', 'c' , \dots\}$
\pause
\item $[Bool] = \{[], [True], [False], [True, False], [False, True], \dots\}$
\pause
\item $[a]$
\end{itemize}

\end{exampleblock}
\end{frame}

\begin{frame}
\frametitle{Les types}
\begin{block}{Construire son type :}
\begin{itemize}
\item Trival = Plus | Minus | Zero
\pause
\item Box a = InABox a
\pause
\item Maybe a = Just a | Nothing
\pause
\item Either a b = Left a | Right b
\end{itemize}
\end{block}

\pause

Just, Nothing, InABox etc portent le doux nom de \emph{constructeur de type}. C'est aussi le cas de \emph{[]}.
\end{frame}

\subsection{Fonctions}
\begin{frame}
\frametitle{Les fonctions}
\begin{block}{}
Ce sont les traitements que l'on peut implémenter.
\end{block}
\begin{center}
\includegraphics[scale=0.2]{fct.png}
\end{center}
\pause
\begin{block}{}
Une fonction ne lance pas de fusé.
\end{block}
\end{frame}


\begin{frame}
\frametitle{Les fonctions}
\begin{block}{Une fonction a aussi un type : \emph{a -> b}}
\begin{itemize}
\item floor   :: Float -> Int
\item (+2):: Int -> Int
\pause
\item id      :: a -> a
\item map :: (a -> b) -> [a] -> [b]
\end{itemize}
\end{block}

\end{frame}


\begin{frame}
\frametitle{Les fonctions}
\begin{block}{Ça se compose}
\begin{itemize}
\item f1 :: a -> b
\item f2 :: b -> c
\item f2 . f1 :: a -> c
\pause
\item . :: (a -> b) -> (b -> c) -> (a -> c)
\end{itemize}
\end{block}
\pause

La collection de tous les types forme une catégorie. Les flèches sont les fonctions implémentables. On l’appelle la catégorie des types.

\end{frame}

\subsection{Foncteurs}


\begin{frame}
\frametitle{Les foncteurs}
\begin{block}{Un foncteur F agit sur les types ...}
\begin{itemize}
\item a => F a
\end{itemize}
\end{block}
\begin{exampleblock}{}
\begin{itemize}
\item a => Maybe a
\item a => [a]
\end{itemize}
\end{exampleblock}

\pause

\begin{block}{... et sur les fonctions}
\begin{itemize}
\item a -> b => F a -> F b
\end{itemize}
\end{block}

\begin{exampleblock}{}
\begin{itemize}
\item fmap (+2) :: F Int -> F Int
\item fmap id :: F a -> F a
\end{itemize}
\end{exampleblock}
\end{frame}

\begin{frame}
\frametitle{Donnée dans un contexte}

\begin{block}{}
Un foncteur permet de passer d'un monde (les types a) vers un autre (les types F a).
\end{block}

\begin{center}
\includegraphics[scale=0.3]{just3.png}
\end{center}
TODO : Remplacer l'image par a -> F a

\end{frame}

\begin{frame}
\frametitle{Functorial mapping}
On ne peut plus appliquer la fonction telle quelle les diagrammes suivant commutent :

\begin{center}
\includegraphics[scale=0.3]{wrong_type.png}
\end{center}
\end{frame}

\begin{frame}
\frametitle{Functorial mapping}
Mais le foncteur nous donne une nouvelle flèche.
\begin{center}
\includegraphics[scale=0.19]{f_fct.png}
\end{center}
\end{frame}



\begin{frame}
\frametitle{Dura lex sed lex}
\begin{alertblock}{Un foncteur doit respecter des lois}
\begin{itemize}
\item fmap id = id
\item fmap (p . q) = (fmap p) . (fmap q)

\end{itemize}
\end{alertblock}

\pause
Un foncteur est un endofoncteur de la catégorie des types.
\end{frame}

\subsection{Monades}

\begin{frame}

\begin{center}\textbf Monades\end{center}

TODO : Funny picture here.
\end{frame}

\begin{frame}
\frametitle{Donnée dans un contexte}

\begin{block}{}
Une monade place une valeur dans un contexte.
\end{block}

\begin{center}
\includegraphics[scale=0.3]{just3.png}
\end{center}

\begin{exampleblock}{}
L'exemple de Maybe : \verb!Just 3!
\end{exampleblock}
\end{frame}

\begin{frame}
\frametitle{Donnée dans un contexte}

\begin{block}{}
Un contexte peut aussi ne pas contenir de valeur.
\end{block}

\begin{center}
\includegraphics[scale=0.3]{nothing.png}
\end{center}
\begin{exampleblock}{}
L'exemple de Maybe : \verb!Nothing!
\end{exampleblock}
\end{frame}

\begin{frame}
\frametitle{Placer une donnée dans un contexte}

\begin{block}{L'opérateur \emph{pure}}
\begin{center}
\verb!pure :: a -> F a!
\end{center}
\end{block}

\begin{exampleblock}{Quelques cas particuliers}
	\begin{itemize}
		\item Just
		\item (: [])
		\item Right
	\end{itemize}
\end{exampleblock}
\end{frame}


\begin{frame}
\frametitle{Un traitement qui peut échouer}

\begin{center}
\includegraphics[scale=0.5]{failure.png}
\end{center}
\begin{exampleblock}{}
Une fonction de type \emph{Int -> Maybe Int}.
\end{exampleblock}
\end{frame}

\begin{frame}
\frametitle{Composer des traitements avec échec}
\begin{block}{}
Comment composer \verb!f :: a -> M b! et \verb!g :: b -> M c! ?
\end{block}
\pause
\begin{block}{}
Si $M$ est un foncteur, on peut composer
\verb!f :: a -> M b! avec \verb!fmap g :: M b -> M (M c)!.
\end{block}
\pause
\begin{block}{}
Que faire d'un \verb!M (M c)!?
\end{block}

\end{frame}

\begin{frame}
\frametitle{L'opérateur join}
\begin{block}{}
\begin{center}
\verb!join :: M (M a) -> M a!
\end{center}
\end{block}

\begin{center}
\includegraphics[scale=0.2]{join.png}
\end{center}
\pause
\begin{exampleblock}{}
\begin{center}
\verb!join \$ Just (Just 3)!.
\end{center}
\end{exampleblock}
\end{frame}


\begin{frame}
\frametitle{L'opérateur join}
\begin{block}{}
\begin{center}
\verb!join :: M (M a) -> M a!
\end{center}
\end{block}
\medskip
\begin{center}
\includegraphics[scale=0.2]{join_nothing.png}
\end{center}
\medskip
\pause
\begin{exampleblock}{}
\begin{center}
\verb!join \$ Just (Nothing)!.
\end{center}
\end{exampleblock}
\end{frame}



\begin{frame}
\frametitle{L'opérateur \emph{bind}}
\begin{block}{On cherche à définir la composition.}
\verb!(>=>) :: (a -> M b) -> (b -> M c) -> (a -> M c)!
\end{block}

\pause

\begin{block}{Nous avons :}
\begin{itemize}
\item \verb!(fmap g) . f :: a -> M (M c)!
\item \verb!join :: M (M a) -> M a!
\end{itemize}
\end{block}
\pause

\begin{block}{On peut maintenant composer $f$ et $g$.}

\verb!f >=> g $\equiv$ join . (fmap g) . f!.
\end{block}
\end{frame}

\begin{frame}
\frametitle{Récapitulatif}

\begin{block}{Une monade, c'est}

	\begin{itemize}
		\item fmap :: (a -> b) -> (M a -> M b)
		\item pure :: a -> M a
		\item join :: M (M a) -> M a
	\end{itemize}

\end{block}

\end{frame}

\begin{frame}
\frametitle{Dura lex sed lex}
\begin{alertblock}{Une monade doit respecter des lois}
\begin{itemize}
\item \verb!pure . f $\equiv$ (fmap f) . pure!
\item \verb!join . fmap (fmap f) $\equiv$ (fmap f) . join!
\item[] \ 
\item \verb!join . fmap join   $\equiv$ join . join!
\item \verb!join . fmap pure   $\equiv$ join . pure = id!
\end{itemize}
\end{alertblock}
\end{frame}

\begin{frame}
\frametitle{Monades - Catégories}
Une monade $(T, \mu, \eta)$ est la donné d'un
endofoncteur $T : C \rightarrow C$ et de deux
transformations naturelles $\mu : T\circ T \rightarrow T$ et $\eta : 1_C \rightarrow T$ telles que :

\[
\begin{array}{cc}
\xymatrix{
T(T(T(X))) \ar[r]^{T(\mu_X)} \ar[d]_{\mu_{T(X)}} & T(T(X)) \ar[d]^{\mu_X} \\
T(T(X)) \ar[r]_{\mu_X} & T(X) \\
}
&
\xymatrix{
T(X) \ar[r]^{\eta_{T(X)}} \ar[d]_{T(\eta_X)}  \ar@{=}[dr] & T(T(X)) \ar[d]^{\mu_X} \\
T(T(X)) \ar[r]_{\mu_X} & T(X) \\
}
\end{array}
\]
c'est à dire
$\mu \circ T\mu = \mu \circ \mu_T$
et
$\mu \circ T \eta = \mu \circ \eta_T = id_T$.

\pause

Dans notre cas, $C$ est la catégorie des types.
\end{frame}

\begin{frame}
\frametitle{A chaque loi son diagramme}
\begin{alertblock}{\verb!pure! est une T.N.}
\verb!pure . f $\equiv$ (fmap f) . pure!
\end{alertblock}

\begin{block}{}
\[
\xymatrix{
X \ar[r]^{f} \ar[d]_{\eta_X} & Y \ar[d]^{\eta_Y} \\
T(X) \ar[r]_{T(f)} & T(Y) \\
}
\]
\end{block}

\end{frame}

\begin{frame}
\frametitle{A chaque loi son diagramme}
\begin{alertblock}{\verb!join! est une T.N.}
\verb!join . fmap (fmap f) $\equiv$ (fmap f) . join!
\end{alertblock}

\begin{block}{}
\[
\xymatrix{
T(T(X)) \ar[r]^{T(T(f))} \ar[d]_{\mu_{X}} & T(T(Y)) \ar[d]^{\mu_{Y}} \\
T(X) \ar[r]_{T(f)} & T(Y) \\
}
\]
\end{block}

\end{frame}


\begin{frame}
\frametitle{A chaque loi son diagramme}
\begin{alertblock}{Associativité}
\verb!join . fmap join   $\equiv$ join . join!
\end{alertblock}

\begin{block}{}

\[
\xymatrix{
T(T(T(X))) \ar[r]^{T(\mu_X)} \ar[d]_{\mu_{T(X)}} & T(T(X)) \ar[d]^{\mu_X} \\
T(T(X)) \ar[r]_{\mu_X} & T(X) \\
}
\]
\end{block}

\begin{block}{}
\begin{center}
$\mu \circ T\mu = \mu \circ \mu_T$
\end{center}
\end{block}

\end{frame}

\begin{frame}
\frametitle{A chaque loi son diagramme}
\begin{alertblock}{Existence d'un neutre}
\verb!join . fmap pure   $\equiv$ join . pure = id!
\end{alertblock}

\begin{block}{}
\[
\xymatrix{
T(X) \ar[r]^{\eta_{T(X)}} \ar[d]_{T(\eta_X)}  \ar@{=}[dr] & T(T(X)) \ar[d]^{\mu_X} \\
T(T(X)) \ar[r]_{\mu_X} & T(X) \\
}
\]
\end{block}

\begin{block}{}
\begin{center}
$\mu \circ T \eta = \mu \circ \eta_T = id_T$
\end{center}
\end{block}

\end{frame}


\section{Automates Cellulaires}
\begin{frame}
\begin{center}
\textbf{Automates cellulaires}
\end{center}

\begin{center}
%\includegraphics[scale=0.5]{automata_flip.png}
\includegraphics[scale=0.16]{textile_cone.jpg}
\end{center}

\begin{center}Toison d'or\end{center}

\end{frame}

\begin{frame}
\frametitle{Qu'est-ce qu'un automate cellulaire?}
\begin{block}{Un automate cellulaire, c'est :}
	\begin{itemize}
		\item Un nombre fini d'états $S$,
		\item Une grille de cellules,
		\item La notion de voisinage d'une cellule $V_c$,
		\item D'une fonction de transition qui à une cellule associe sont nouvelle état.
	\end{itemize}
\end{block}
\end{frame}

\begin{frame}
\frametitle{Combien d'automates cellulaires différents?}
\begin{block}{On a le choix :}
	\begin{itemize}
	\item De la dimension de la grille,
	\item Des lois,
	\item Du nombres d'états (couleurs),
	\item De la forme du voisinages (boules de rayon r, etc.),
	\item De ne pas être déterministe.
	\end{itemize}
\end{block}
\end{frame}

\begin{frame}
\frametitle{The "Game of Life"}
\begin{center}
\textbf{Jeu de la vie} (J. H. Conway)
\end{center}
\begin{center}
\includegraphics[scale=0.5]{game_of_life.png}
\end{center}
\end{frame}

\begin{frame}
\frametitle{Étude d'un cas : Rule 30}
\begin{center}
\textbf{Rule 30}
\end{center}
\begin{center}
\includegraphics[scale=1]{rule30.png}
\end{center}
\end{frame}

\begin{frame}
\frametitle{La grille}
\begin{block}{La grille de l'automate}
\begin{center}
\includegraphics[scale=0.5]{grid.png}
\end{center}
\end{block}
\begin{itemize}
	\item Une grille 1D
	\item Deux états (Blanc / Noir)
\end{itemize}
\end{frame}

\begin{frame}
\begin{block}{}
	Un voisinage de 3 cellules.
\end{block}
\pause
\begin{block}{Les règles}
	\begin{center}
	\includegraphics[scale=1]{rule30_rule.png}
	\end{center}
\end{block}
\pause
\begin{block}{On peux aussi écrire :}
	\begin{center}
	\begin{tabular}{|l|c|c|c|c|c|c|c|c|}
	\hline
	Ancien état & 111 & 110 & 101 & 100 & 011 & 010 & 001 & 000 \\
	\hline
	Nouvel état	& 0   & 0   & 0   & 1   & 1   & 1   & 1   & 0 \\
	\hline
	\end{tabular}
	\end{center}
\end{block}
\end{frame}

\section{Comonades}

\begin{frame}
Comonades

TODO : Funny comonad picture
\end{frame}

\begin{frame}
\frametitle{C'est le dual d'une monade}

\begin{itemize}
\item extract (copure) (co uinit) :: M a -> a
\item duplicate (cojoin) (co product $\delta$) :: M a -> M (M a)
\end{itemize}
\end{frame}

\begin{frame}
\frametitle{Dura lex sed lex}
\begin{alertblock}{Une comonade doit respecter des lois}
\begin{itemize}
\item \verb!(fmap (fmap f)) . duplicate $\equiv$ duplicate . fmap f!
\item \verb!duplicate . duplicate = fmap duplicate . duplicate!
\item[] \ 
\item \verb!duplicate . duplicate $\equiv$ fmap duplicate . duplicate! (commut)
\item \verb!fmap extract . duplicate $\equiv$ extract . duplicate $\equiv$ id! (co unit)
\end{itemize}
\end{alertblock}
\end{frame}


\begin{frame}
\frametitle{Comonades - Catégories}
Une comonade $(T, \delta, \epsilon)$ est la donné d'un
endofoncteur $T : C \rightarrow C$ et de deux
transformations naturelles $\Delta : T \rightarrow T\circ T$ et $\epsilon : T \rightarrow 1_C$ telles que :

\[
\begin{array}{cc}
\xymatrix{
T(X) \ar[r]^{\Delta_X} \ar[d]_{\Delta_X} & T(T(X)) \ar[d]^{\Delta_{T(X)}} \\
T(T(X)) \ar[r]_{T(\Delta_X)} & T(T(T(X))) \\
}
&
\xymatrix{
T(X) \ar[d]_{\Delta_X} \ar[r]^{\Delta_X}  \ar@{=}[dr] & T(T(X))   \ar[d]^{\epsilon_{T(X)}} \\
T(T(X)) \ar[r]_{T(\epsilon_X)} & T(X) \\
}
\end{array}
\]
c'est à dire
$\Delta_T \circ \Delta = T \Delta \circ \Delta$
et
$T \epsilon \circ \Delta = \epsilon_T \circ \Delta = id$.
\end{frame}

\begin{frame}
\frametitle{A chaque loi son diagramme}
\begin{alertblock}{\verb!extract! est une T.N.}
\verb!f . extract $\equiv$ extract . (fmap f)!
\end{alertblock}

\begin{block}{}
\[
\xymatrix{
X \ar[r]^{f} & Y  \\
T(X) \ar[r]_{T(f)} \ar[u]_{\epsilon_X} & T(Y) \ar[u]^{\epsilon_Y}\\
}
\]
\end{block}

\end{frame}

\begin{frame}
\frametitle{A chaque loi son diagramme}
\begin{alertblock}{\verb!duplicate! est une T.N.}
\verb!(fmap (fmap f)) . duplicate $\equiv$ duplicate . fmap f!
\end{alertblock}

\begin{block}{}
\[
\xymatrix{
T(X) \ar[r]_{T(f)} \ar[d]_{\Delta_{X}} & T(Y) \ar[d]^{\Delta_{Y}} \\
T(T(X)) \ar[r]^{T(T(f))} & T(T(Y)) \\
}
\]
\end{block}

\end{frame}


\begin{frame}
\frametitle{A chaque loi son diagramme}
\begin{alertblock}{Coassociativité}
\verb!duplicate . duplicate = fmap duplicate . duplicate!
\end{alertblock}

\begin{block}{}

\[
\xymatrix{
T(X) \ar[r]^{\Delta_X} \ar[d]_{\Delta_X} & T(T(X)) \ar[d]^{\Delta_{T(X)}} \\
T(T(X)) \ar[r]_{T(\Delta_X)} & T(T(T(X))) \\
}
\]
\end{block}

\begin{block}{}
\begin{center}
$\Delta_T \circ \Delta = T\Delta \circ \Delta$
\end{center}
\end{block}

\end{frame}

\begin{frame}
\frametitle{A chaque loi son diagramme}
\begin{alertblock}{Existence d'une counité}
\verb!extract . duplicate = fmap extract . duplicate = id!
\end{alertblock}

\begin{block}{}
\[
\xymatrix{
T(X) \ar[d]_{\Delta_X} \ar[r]^{\Delta_X}  \ar@{=}[dr] & T(T(X))   \ar[d]^{\epsilon_{T(X)}} \\
T(T(X)) \ar[r]_{T(\epsilon_X)} & T(X) \\
}
\]
\end{block}

\begin{block}{}
\begin{center}
$\epsilon_T \circ \Delta = T \epsilon \circ \Delta = id_T$
\end{center}
\end{block}

\end{frame}


\end{document}
